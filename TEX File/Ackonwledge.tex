\documentclass[hyperref, UTF8]{ctexart}

\usepackage{geometry}
\geometry{a4paper}

\bibliographystyle{plain}

\usepackage{ulem}     % 删除线指令\sout{}
\usepackage{faktor}   % 斜分数指令\faktor{}{} (数学模式)
\usepackage{nicefrac} % 斜分数指令\nicefrac{}{} (文本公式)
\usepackage{amsopn}   % 新算符指令\DeclareMathOperator{command}{text}
\usepackage{xfrac}	  % 斜分式指令\sfrac
\usepackage{mathtools}
\usepackage{amstext}  % 数学文本指令\text{}
\usepackage{color}    % 文本颜色指令\color
\usepackage{amsmath}  % 数学环境宏包
\usepackage{amsthm}   % 数学环境宏包2
\usepackage{graphicx} % 插入图片命令\includegraphics
\usepackage{amsfonts} % \mathbb命令用
\usepackage{mathrsfs} % \mathscr命令用
\usepackage{hyperref} % 加载为电子文档提供目录标签和超链接功能的hyperref宏包
\hypersetup{
	colorlinks = true,
	linkcolor = blue
}


\newtheorem*{Exam}{例}   % 例环境Exam
\newtheorem{Defi}{定义} % 定义环境Defi
\newtheorem*{Note}{注}   % 注环境Note
\newtheorem{Theo}{定理} % 定理环境Theo
\newcommand\dd{\mathrm{d}}
\DeclareMathOperator{\fE}{fE}
\DeclareMathOperator{\sinn}{sinn}
\DeclareMathOperator{\arcsinh}{arcsinh}

\title{致谢}



\begin{document}
	
\maketitle

感谢所有帮我编写这本书的人. 我特别感激Fritz Waitz, 他的评论, 想法与纠正让本书质量大大改善. 我十分感谢Arne Becker 和Daniel Hilpert, 感谢他们无价的建议,意见与细致的校对. 感谢Robert Sadlier对我英文的帮助以及Jakob Karalus的解释.

我还想感谢与我有许多见解深刻的讨论的Marcel K$\ddot{o}$pke, 感谢Silvia Schwichtenberg和 Christian Nawroth 的支持.

最后, 我亏欠最多的是我的父母, 他们支持着我, 教导我知识高于一切.

如果发现文中的错误, 我非常希望你能够寄一封短邮件到 errors@jakobschwichtenberg.com. 勘误表的地址是\url{http://physicsfromsymmetry.com/errata}.
\end{document}



