%!TEX encoding = UTF-8 Unicode

%----------------------------------------------------------------------------------------
%	CHAPTER 6
%----------------------------------------------------------------------------------------

\chapterimage{chapter_head_1.pdf} % Chapter heading image

\chapter[自由场理论]{Free Theory 自由场理论}\label{chap6}

在这章中,我们建立自由的体系的物理理论,也就是说无相互作用的情况下的对称性给出的场\mpar{尽管我们这里针对的是场,后面会看到这里得到的方程同样可以用来处理粒子}。我们会
\begin{itemize}
\item 利用Lorentz群的$(0,0)$表示给出Klein-Gordon方程
\item 利用Lorentz群的$(\tfrac{1}{2},0)\oplus(0,\tfrac{1}{2})$表示给出Dirac方程
\item 利用Lorentz群的矢量$(\tfrac{1}{2},\tfrac{1}{2})$表示给出Proca方程,在无质量极限下退化到著名的Maxwell方程组
\end{itemize}

\section[Lorentz协变性与不变量]{Lorentz Covariance and Invariance Lorentz协变性与不变量}\label{sec6.1}

在下面的章节中,我们会得到粒子物理的标准模型- 这是我们拥有的最好的物理理论 -中的运动方程。我们期待这些房产在所有的惯性系中看起来都一样,因为如果不这样的话,我们就会对每一个可能的参考系都有各自的方程。狭义相对论指出并没有一个特别的参考系,所以这一点是没有意义的。这个术语被称为{\bf Lorentz协变性}。一个Lorentz协变的物体是说它按照某一种给定的Lorentz群来进行变换。比如说,矢量$A_\mu$,通过$\left(\frac{1}{2},\frac{1}{2}\right)$表示来进行变换,因此是Lorentz协变的;这其实是说,$A_\mu\to A'_{\mu'}$,但实际上这两个指代同一个东西,并不是完全不同的。另一方面,比如说,$A_1+A_3$就不是Lorentz协变的,因为它不按照Lorentz群的某种表示来变换。但这并不是说我们不知道它是如何变换的,因为它的变换规律从$A_\mu$的变换规律里面可以简单地得出,但是在不同的惯性系中它看起来完全不一样。在一个``推动''的参考系中它看起来或许像$A_2+A_4$。只涉及到Lorentz协变的东西的方程被称为Lorentz协变方程,比如
\[A_\mu+7B_\mu+C_\nu A^\nu D_\mu = 0 \]
就是一个Lorentz协变方程,因为在另一个参考系中,它看起来就是
\[A_\mu+7B_\mu+\underbrace{C_\nu A^\nu}_{\mathclap{=\text{一个Lorentz标量,根据$(0,0)$表示来进行变换}}} D_\mu = A'_\mu+7B'_\mu+C_\nu A^\nu D'_\mu = 0 \]
我们看到它看起来是一样的。一个只有部分是这样的东西的方程,一般来说并不是Lorentz协变的,从而在另一个惯性系里面就会不一样。

为了确定我们只使用Lorentz{\bf 协变}的方程,我们需要要求作用量$S$是Lorentz{\bf 不变}的。这是说它只能包含那些在参考系变换下保持不变的成分。换句话说,作用量里面只可以包含Lorentz变换下不变的部分。从作用量$S$中我们可以得到运动方程\mpar{回忆一下我们保持作用量最小以及得来的欧拉-拉格朗日方程,就是对系统的运动方程}。如果现在$S$依赖于参考系,也就是说得到的运动方程不再是Lorentz协变的了。

就像我们之前一章讨论的那样,我们可以用更严格的约束条件来要求拉格朗日量应该是不变的,因为这样的话作用量也就肯定是不变的了。

\section[Kelin-Gordon 方程]{Kelin-Gordon Equation Kelin-Gordon 方程}\label{sec6.2}

我们现在开始考虑最简单的情况:标量,其按照Lorentz群的$(0,0)$表示变化。我们需要找一个对应的拉格朗日量来确定标量的运动方程。一个符合我们限制的一般的拉格朗日量第\mpar{在\ref{sec4.2}节中有所讨论:我们只考虑$0, 1$和$2$阶的$\Phi$。只考虑最低可能的导数项的原因在后面就会说明。}是:

\begin{align}
\mathcal{L} = A\Phi^0+V\Phi+C\Phi^2+D\partial_\mu\Phi+E\partial_\mu\Phi\partial^\mu\Phi+F\Phi\partial_\mu\Phi
\end{align}
首先,需要注意我们考虑的是拉格朗日量密度$\mathcal{L}$而不是$L$本身,然后我们的物理理论可以从作用量
\begin{align}
S = \int dx\mathcal{L}
\end{align}
得出,其中$dx$可以理解为对时间和空间的积分。因此,诸如$\Phi\partial_\mu\partial^\mu\Phi$的项实际是没必要的,因为它等效于$\partial_\mu\Phi\partial^\mu\Phi$,从分部积分就可以直接地看到这一点\mpar{很直接的,边界项会被小区,因为在远处场很小。值得注意的是,这是由于我们物理里面的速度是有上限的(第\ref{sec2.3}节)。因此,很远处的场对于有限远的$x$位置的场不会有影响}。

不仅如此,Lorentz{\bf 不变形}要求拉格朗日量必须是一个标量。因此,所有奇数阶的$\partial_\mu$,比如$\partial_\mu\Phi$是禁止的。值得一提的是,如果常数,比如$a, c$,有一个Lorentz指标的话,这说明这些常数是一个$4$-矢量,标定了时空的方向从二破坏了空间的各向同性的要求。我们实际上可以忽略常数项,也就是$A=0$,因为我们的物理理论需要从欧拉-拉格朗日方程中得到,因此一个常数项对运动方程没有影响\mpar{考虑\eqref{eq4.10}:$\frac{\partial{\mathcal{L}}}{\partial\Phi}-\partial_\mu\left(\frac{\partial{\mathcal{L}}}{\partial(\partial_\mu\Phi)}\right)$,因此$\mathcal{L}\to\mathcal{L}+A$中的常数项$A$并不改变任何东西:$\frac{\partial({\mathcal{L}}+A)}{\partial\Phi}-\partial_\mu\left(\frac{\partial({\mathcal{L}}+A)}{\partial(\partial_\mu\Phi)}\right) = \frac{\partial{\mathcal{L}}}{\partial\Phi}-\partial_\mu\left(\frac{\partial{\mathcal{L}}}{\partial(\partial_\mu\Phi)}\right)$}。

\subsection[复Kelin-Gordon场]{Complex Kelin-Gordon Field 复Kelin-Gordon场}\label{sec6.2.1}

\section[Dirac方程]{Dirac Equation Dirac 方程}\label{sec6.3}


\section[Proca方程]{Proca Equation Proca 方程}\label{sec6.4}







