%!TEX root = ..\main.tex
%!TEX encoding = UTF-8 Unicode
%----------------------------------------------------------------------------------------
%	CHAPTER 4
%	Translator: SI(= Surgam Identidem)
%	Proofreader: lh1962
%----------------------------------------------------------------------------------------

\chapterimage{chapter_head_1.pdf} % Chapter heading image






\chapter[经典力学]{Classical Mechanics \quad 经典力学}
\label{chap10}
本章探究量子力学与经典力学之间的联系,我们会看到动量算符期望值对时间的导数正是 Newton 第二定律(经典力学的基础之一)的形式。

力学量算符$\hat{\mathcal{O}}$的期望值为(\eqref{equ8.14}式):
\[
    \langle \hat{\mathcal{O}} \rangle = \int d^3 x \Psi^* \hat{\mathcal{O}} \Psi
\]
势场$V$中单粒子的 Schr\"{o}dinger 方程(\eqref{equ8.22}式):
\[
\begin{split}
    (i \frac{d}{dt} + \frac{\nabla^2}{2m}) \Psi - V \Psi = 0 \\
    \to i \frac{d}{dt} \Psi = \underbrace{ \left( -\frac{\nabla^2}{2m} + V \right) }_{=: H} \Psi \\
    \to \frac{d}{dt} \Psi = \frac{1}{i} H \Psi \\
    \underbrace{ \to }_{\mathclap{H^\dag = H}} \frac{d}{dt} \Psi^* = - \frac{1}{i} \underbrace{\Psi^*}_{\mathclap{\Psi^\dag = \Psi^*}} H
\end{split}
\]

期望值对时间的导数为:
\[
    \frac{d}{dt} \langle \hat{\mathcal{O}} \rangle = \int d^3 x \left( \left( \frac{d}{dt} \Psi^* \right) \hat{\mathcal{O}} \Psi + \Psi^* \left( \frac{d}{dt} \hat{\mathcal{O}} \right) \Psi + \Psi^* \hat{\mathcal{O}} \left( \frac{d}{dt} \Psi \right) \right)
\]
对大部分算符都有$\frac{d}{dt} \hat{\mathcal{O}} = 0$,例如动量$\hat{\mathcal{O}} = \hat{\vec{p}} = -i \vec{\nabla} \neq \hat{\mathcal{O}} (t)$. 我们还利用Schr\"{o}dinger方程来重写上式中波函数及其共轭对时间的导数项:
\begin{align}
    \frac{d}{dt} \langle \hat{\mathcal{O}} \rangle  &= \int d^3 x \left( \left( - \frac{1}{i} \Psi^* H \right) \hat{\mathcal{O}} \Psi + \Psi^* \hat{\mathcal{O}} \left( \frac{1}{i} H \Psi \right) \right) \notag \\
    &= \frac{1}{i} \int d^3 x \big( ( -\Psi^* H) \hat{\mathcal{O}} \Psi + \Psi^* \hat{\mathcal{O}} (H\Psi) \big) \notag \\
    &= \frac{1}{i} \int d^3 x \Psi^* [\hat{\mathcal{O}}, H] \Psi \notag \\
\label{equ10.1}
    &= \frac{1}{i} \langle [\hat{\mathcal{O}}, H] \rangle
\end{align}
上式是著名的{\bfseries Ehrenfest 定理}。比如,考虑$\hat{\mathcal{O}} = \hat{p}$,$H = \frac{p^2}{2m} + V$:
\begin{align}
    \frac{d}{dt} \langle \hat{p} \rangle &= \frac{1}{i} \langle [\hat{p}, H] \rangle \notag \\
    &= \frac{1}{i} \left\langle [ \hat{p}, \frac{\hat{p}^2}{2m} + V] \right\rangle \notag \\
    &= \frac{1}{i} \left\langle \underbrace{ [\hat{p}, \frac{\hat{p}^2}{2m}] }_{ = 0} + [\hat{p}, V] \right\rangle \notag \\
    &= \frac{1}{i} \langle [\hat{p}, V] \rangle \notag \\
    &= \frac{1}{i} \int d^3 x \Psi^* [\hat{p}, V] \Psi \notag \\
    &= \frac{1}{i} \int d^3 x \Psi^* \hat{p} V \Psi - \frac{1}{i} \int d^3 x \Psi^* V \hat{p} \Psi \notag \\
    &= \frac{1}{i} \int d^3 x \Psi^* (-i \nabla) V\Psi - \frac{1}{i} \int d^3 x \Psi^* V (-i \nabla) \Psi \notag \\
    &\underbrace{=}_{\text{乘积法则}}  - \int d^3 x \Psi^* (\nabla V) \Psi - \int d^3 x \Psi^* V \nabla \Psi + \int d^3 x \Psi^* V \nabla \Psi \notag \\
    &= - \int d^3 x \Psi^* (\nabla V) \Psi \notag \\
\label{equ10.2}
    &= \langle - \nabla V \rangle = \langle F \rangle
\end{align}
上式意味着动量期望对时间的导数等于负的势能梯度(也就是力)的期望值,这正是{\bfseries Newton 第二定律}\mpar{在\ref{sec4.5}节由 Noether 定理解释守恒量的时候直接用过这个式子(没推导),这里正式把它推倒(无误)啦。}。它可以计算宏观物体的运动轨迹。历史上这个关于力的定律是从实验中总结的唯象定律,作用于一个物体上的力在方程的右端线性相加。定义唯象的物理量{\bfseries 动量} $p_{\text{mak}} = mv$, Newton 第二定律可写为$\frac{d}{dt} p_{\text{mak}} = \frac{d}{dt} mv$,当物体的质量不变时等号右侧即为$m \frac{d}{dt} v$。速度就是坐标对时间的导数\mpar{换句话说,速度$v = \frac{d}{dt} x(t) = \dot{x}(t)$是物体坐标的时间变化率,同理,加速度$a = \frac{d}{dt} \frac{d}{dt} x(t) = \ddot{x}(t)$ 是速度的时间变化率。},综上可得:
\begin{equation}
\label{equ10.3}
    m \frac{d^2}{d t^2} x = F_1 + F_2 + \dots
\end{equation}
解这个微分方程就得到物体的运动轨迹$x = x(t)$。下一章有一个关于经典力学这方面的例子。

\section[相对论力学]{Relativistic Mechanics \quad 相对论力学}
\label{sec10.1}
拉格朗日形式提供了有关经典力学相当不同的一种视角。描述一个粒子的运动需要一个方程。本书总是假设正确的方程是使{\itshape 某物}取极小值而导出的。第\ref{chap4}章中已经讲过它必须是Lorentz变换下的不变量,否则不同惯性系的运动规律会不同。

在狭义相对论中有一个现成的Lorentz不变量:时空间隔(在\ref{sec2.1}节推导)
\begin{equation}
\label{equ10.4}
    (ds)^2 = (c d\tau)^2 = (c dt)^2 - (dx)^2 - (dy)^2 - (dz)^2
\end{equation}
其中$\tau$是固有时(见\ref{sec2.2}节)。$(ds)^2$的平方根自然是不变量,由此可以构造一个用来取极小值,并且形式上最简单的量:
\begin{equation}
\label{equ10.5}
    S = \int C d \tau,\  \text{其中}C\text{是常数,} d\tau = \frac{1}{c} \sqrt{(cdt)^2 - (dx)^2 - (dy)^2 - (dz)^2}
\end{equation}
正确的常数应当是$C = -mc^2$,即我们要使
\begin{equation}
\label{equ10.6}
    S = -mc^2 \int d\tau
\end{equation}
取极小值。

简明起见,只考虑一维情形:
\begin{align}
    d \tau &= \frac{1}{c} \sqrt{ (c dt)^2 - (dx)^2} = \frac{1}{c} \sqrt{ (cdt)^2 \left( 1 - \frac{(dx)^2}{ c^2 (dt)^2 } \right) } \notag \\
\label{equ10.7}
    &= \frac{1}{c} (c dt) \sqrt{ 1 - \frac{1}{c^2} \left( \frac{dx}{dt} \right)^2 } \underbrace{=}_{\mathclap{\frac{dx}{dt} = \dot{x}, \text{即为粒子速度.}}} dt \sqrt{ 1 - \frac{\dot{x}^2}{c^2} }.
\end{align}
带入\eqref{equ10.6}式可得:
\begin{equation}
\label{equ10.8}
    S = \int \underbrace{-mc^2 \sqrt{1 - \frac{\dot{x}^2}{c^2} }}_{\equiv \mathcal{L}} dt
\end{equation}

像以前那样,通过将$\mathcal{L}$带入欧拉-拉格朗日方程\eqref{equ4.7}式就可得到使$S$取极小值的方程:
\begin{align}
    \frac{\partial \mathcal{L}}{\partial x} - \frac{d}{dt} \left( \frac{\partial \mathcal{L}}{\partial \dot{x}} \right) &= 0 \notag \\
    \to \underbrace{ \frac{\partial}{\partial x} \left( -mc^2 \sqrt{1 - \frac{\dot{x}^2}{c^2}} \right) }_{ = 0} - \frac{d}{dt} \left( \frac{\partial}{\partial \dot{x}} \left( -mc^2 \sqrt{1 - \frac{ \dot{x}^2}{c^2} } \right) \right) &= 0 \notag \\
\label{equ10.9}
    \to c^2 \frac{d}{dt} \left( \frac{ -m \frac{\dot{x}}{c^2} }{\sqrt{1 - \frac{ \dot{x}^2}{c^2} } } \right) = 0 \rightarrow \frac{d}{dt} \left( \frac{m \dot{x}}{\sqrt{1 - \frac{\dot{x}^2}{c^2}}} \right) &= 0
\end{align}
这正是相对论下的{\bfseries 自由}粒子的运动方程。若粒子在外势场$V(x)$中运动,则在拉格朗日量中附上势能项就可以了\mpar{实际上添加的是$-V(x)$而非$V(x)$,这是为了保证待会儿根据 Noether 定理计算体系的能量时,势能项为$+ V ( x )$.}:
\begin{equation}
\label{equ10.10}
    \mathcal{L} = -mc^2 \sqrt{1 - \frac{\dot{x}^2}{c^2} } - V(x)
\end{equation}
这个拉格朗日量带入欧拉-拉格朗日方程得到:
\begin{equation}
\label{equ10.11}
    \frac{d}{dt} \left( \frac{m \dot{x}}{ \sqrt{1 - \frac{\dot{x}^2}{c^2} } } \right) = - \frac{dV}{dx} \equiv F
\end{equation}
注意在非相对论极限$\dot{x} \ll c$之下$\sqrt{1 - \frac{\dot{x}^2}{c^2} } \approx 1$,于是上式化为Newton第二定律形式(\eqref{equ10.3})式。

\section[非相对论力学的拉格朗日量]{The Lagrangian of Non-Relativistic Mechanics \quad 非相对论力学的拉格朗日量}
\label{sec10.2}
上一节导出的拉格朗日量在非相对论极限下的行为非常值得讨论,非相对论极限指粒子速度远小于光速的情形:$\dot{x} \ll c$. 由此可利用Taylor公式\mpar{Taylor公式的简介见附录B.3.}:
\begin{equation}
\label{equ10.12}
    -mc^2 \sqrt{1 - \frac{\dot{x}^2}{c^2}} = -mc^2 \left( 1 - \frac{1}{2} \frac{\dot{x}^2}{c^2} + \dots \right).
\end{equation}
在$\dot{x} \ll c$的极限下可以忽略$\dot{x}/c$的高阶小量,于是拉格朗日量化为:
\begin{equation}
\label{equ10.13}
    \mathcal{L} = -mc^2 + \frac{1}{2} m \dot{x}^2 - V(x)
\end{equation}
前面说过,拉格朗日量中$-mc^2$这样的常数对运动方程没有影响,因此{非相对论力学的拉格朗日量}为:
\begin{equation}
\label{equ10.14}
    \mathcal{L} = \frac{1}{2} m \dot{x}^2 - V(x)
\end{equation}
在外势场为零,即$V(x) = 0$的情形下,$\mathcal{L} = \frac{1}{2} m \dot{x}^2$正是\ref{sec4.5.1}节中由Noether定理导出守恒量所用的拉格朗日量。把\eqref{equ10.14}式的拉格朗日量带入欧拉-拉格朗日方程\eqref{equ4.7}式可得:
\begin{align}
    \frac{\partial \mathcal{L}}{\partial x} - \frac{d}{dt} \left( \frac{\partial \mathcal{L}}{\partial \dot{x}} \right) = 0 \notag \\
    \to \frac{\partial}{\partial x} \left( \frac{1}{2} m \dot{x}^2 \right) - \frac{d}{dt} \left( \frac{\partial}{\partial \dot{x}} \left( \frac{1}{2} m \dot{x}^2 - V(x) \right) \right) = 0 \notag \\
    \to -\frac{\partial}{\partial x} V(x) - \frac{d}{dt} (m \dot{x}) = 0 \notag \\
\label{equ10.15}
    \to \frac{d}{dt} (m \dot{x}) = - \frac{\partial}{\partial x} V(x)
\end{align}
这正是本章开头导出的Newton第二定律\eqref{equ10.3}式。
