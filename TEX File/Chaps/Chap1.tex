%!TEX encoding = UTF-8 Unicode

%----------------------------------------------------------------------------------------
%	CHAPTER 1
%----------------------------------------------------------------------------------------

\chapterimage{chapter_head_1.pdf} % Chapter heading image

\chapter{Introduction 简介}\label{chap1}

\section{What we Cannot Derive 得不到的事情}

在我们开始讲我们能从对称性里面了解到什么之前,我们首先澄清一下我们需要在我们的理论中人为的加一些什么东西。首先,目前没有任何理论可以得到自然界的常数。这些常数需要从实验中提取出来,比如各种相互作用的耦合常数啊,基本粒子的质量啊这种的。

除了这些,我们还有一些东西解释不了:{\bf 数字$3$}。这不是术数的那种神秘主义的东西,而是我们不能解释所有的直接与数字$3$相联系的限制。比如:

\begin{itemize}
\item 对应三种标准模型描述的基本作用力有三种规范理论\mpar{如果你不理解这个简介中的某些名词,比如规范理论或者二重覆盖,不需要太过担心。本书将会详尽的解释,在这里提到只是为了完整性。}。这些力是由分别对应于对称群$U(1), SU(2)$和$SU(3)$的规范理论描述的。为什么没有对应$SU(4)$带来的基本作用力?没人知道!
\item 轻子有三代,夸克也有三代。为什么没有第四代?我们只能从实验\mpar{比如,现在宇宙中元素的丰度是依赖于代的数量的。更进一步,对撞机实验中有对此的很强的证据。(见Phys. Rev. Lett. 109, 241802)}中知道没有第四代。
\item 我们只在拉格朗日量里面包含$\Phi$的最低三阶$(\Phi^0, \Phi^1, \Phi^2)$,其中$\Phi$指代一些描述我们的物理系统的东西,是个通称,而这个拉格朗日量则是被我们用来得到我们的描述自由(=无相互作用)场/粒子的靠谱的理论的。
\item 我们只用三个基本的Poincare群双覆盖的表示,分别对应自旋$0, \tfrac{1}{2}$和$1$。没有基本粒子的自旋是$\tfrac{3}{2}$。
\end{itemize}

在现代的理论中,这些是我们必须手动增加的假定。我们从实验上知道这些假定是正确的,但是目前为止我们没有更深刻的原理告诉我们为什么我们需要到$3$就停。

除此之外,还有两件事情我们没法从对称性中得到,但是他们对于一个严谨的理论来说有时必须被考虑到的:

\begin{itemize}
\item 我们只允许在拉格朗日量中引入尽可能低阶的非平庸的微分算符$\partial_\mu$。对于一些理论,我们使用一阶的微分算符$\partial_\mu$,而另一些理论Lorentz不变形禁止了一阶导数,从而二阶导数$\partial_\mu\partial^\mu$是最低阶的可能的非平庸阶项。除此之外我们就再也得不到一个合理的理论了。存在高阶导数项的理论没有下界,这导致能量可以是一个任意大小的负值;因此,这些理论中的态总可以变到能量更低的态,从而永远不会稳定。
\item 我们处于类似的原因,我们可以说如果半整数自旋的粒子和整数自旋的粒子拥有完全一样的行为的话,宇宙中就不会有稳定的物质。因此,这两者必然有某些{\it 东西}不一样,而我们没得选,只有一种可能而且合理的选择\mpar{我们在最开始的量子场论里使用反对易子而不是对易子,从而防止我们的理论变成没有下界的理论。}是正确的。这引出了半整数自旋粒子的Fermi-Dirac统计的概念和整数自旋粒子的Bose-Einstein统计的概念。半整数自旋的粒子从而通常被称为Fermion,它们中永远不存在两个粒子处在完全一样的态上。而与之相反的,这种情况对整数自旋的粒子 -- 通常被称为Boson -- 是可能的。
\end{itemize}

最后呢,我们提一下剩下的一个我们不能从这本书的其他理论中得到的东西:{\bf 引力}。当然,实际上,大名鼎鼎的广义相对论就是优美而准确的描述引力的理论;然而这个理论与其他理论完全不一样,超出了本书的研究范围。而尝试将引力问题划入相同框架下的量子引力理论仍待完善:目前没有人能够成功得出它。除此之外,在最后一章我们会做一些对引力的一些评述。

\section{Book Overview 全书概览}

\begin{center}
  \makebox[\textwidth][c]{\quad\quad\quad\quad
\small\xymatrix{
& \underset{\overset{|}{\text{不可约表示}}}{\text{Poincare群的双覆盖}}\ar[d]  &\\
&\ar[dl]\ar[d]\ar[dr] &\\
(0,0): \text{自旋$0$表示}\ar[d]_{\text{作用}}^{\text{在}}&(\tfrac{1}{2},0)\oplus(0,\tfrac{1}{2}): \text{自旋$\tfrac{1}{2}$表示} \ar[d]_{\text{作用}}^{\text{在}}&(\tfrac{1}{2},\tfrac{1}{2}): \text{自旋$1$表示}\ar[d]_{\text{作用}}^{\text{在}}\\
\text{标量}\ar[d]_{\text{保证拉格朗日量}}^{\text{是(对称变换)不变的}}&\text{旋量} \ar[d]_{\text{保证拉格朗日量}}^{\text{是(对称变换)不变的}}&\text{矢量}\ar[d]_{\text{保证拉格朗日量}}^{\text{是(对称变换)不变的}}\\
\text{自由的自旋$0$体系的拉格朗日量}\ar[d]_{\text{欧拉-拉格朗}}^{\text{日方程}}&\text{自由的自旋$\tfrac{1}{2}$体系的拉格朗日量} \ar[d]_{\text{欧拉-拉格朗}}^{\text{日方程}}&\text{自由的自旋$1$体系的拉格朗日量}\ar[d]_{\text{欧拉-拉格朗}}^{\text{日方程}}\\
\text{Klien-Gordon方程}&\text{Dirac方程}&\text{Proka方程}
}}
\end{center}

这本书使用{\bf 自然单位制},也就是说Planck常数$\hbar = 1$,光速$c=1$。这是基本理论中使用的惯例,它免除了很多不必要的笔墨。而对于应用来说呢,这些常数需要被再一次的加上去从而回到标准的SI单位制。

{\bf 狭义相对论}的基本假设是我们的起始点;它们是:在所有的惯性参考系--一些相互之间的速度保持恒定的参考系--中,光的速度不变,为$c$;而且所有的惯性参考系中的物理是一样的。

满足这些对称性的所有的变换构成的集合叫做{\bf Poincare群}。





