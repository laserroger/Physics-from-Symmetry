%!TEX encoding = UTF-8 Unicode

%----------------------------------------------------------------------------------------
%	CHAPTER 1
%----------------------------------------------------------------------------------------

\chapterimage{chapter_head_1.pdf} % Chapter heading image

\chapter[测量]{Measuring Nature 测量}\label{chap5}

我们现在已经发现了对称性与守恒量之间的关系了,那么我们接下来就可以利用这种联系。用更专业的话说,Noether定理建立了对称变换的生成元与一个守恒量之间的联系。我们本章将利用这种联系。

守恒量常常被物理学家们用来描述物理系统,因为无论这个系统经历了怎样复杂的变化,守恒量都是一样的。比如,物理学家们为了描述一个实验会使用动量,能量或角动量。Noether定理提示我们了一个方向和一个至关重要的想法:我们在挑选哪些量我们用来描述自然的时候,我们实际上在看伴随着其对应的生成元:

\begin{align}
\text{物理量}\Rightarrow\text{对应对称性的生成元}
\end{align}

我们会看到,这种选择会自然地引导我们给出给出量子理论。

\section[量子力学中的算符]{Operators of Quantum Mechanics 量子力学中的算符}\label{sec5.1}

{\it 惯例上用一个尖帽子$\hat{O}$来表示一个算符}

拉格朗日量在空间平移操作的生成元的作用下不变导致我们得出动量守恒。因此,我们可以得到
\[\text{动量}\hat{p}_i\to\text{空间平移操作的生成元} - i\partial_i \]

类似的,时间平移的生成元带来的不变形给出我们能量守恒,也就是
\[\text{能量}\hat{E}\to\text{时间平移操作的生成元} - i\partial_0 \]

没有关于``位置守恒''的对称性,所以位置并没有伴随生成元,我们有\mpar{或者说,我们可以观察守恒量与对应的推动下的不变性。注意到我们在\ref{sec4.6}节中得到的非相对论性Galilei推动下的守恒量是从非相对论性的拉格朗日量得出来的。统一,我们可以对相对论性拉格朗日量做一样的事情,得到守恒量$i\vec{p}-\vec{x}E$。相对论性能量由$E=\sqrt{m^2+p^2}$给出。在相对论性极限$c\to\infty$下,我们有$E\approx m$,并且,Lorentz推动下的守恒量回到我们的道德Galilei推动的结果。因此,粒子理论的守恒量为$M_i = (tp_i-x_iE)$。推动的生成元(见\eqref{eq3.240},其中$K_i = M_{0i}$)为$K_i = i(x^0\partial_i-x_i\partial_0)$。比较两者,其中$x_0=t$,给出$M_i = (tp_i-x_iE)\leftrightarrow K_i = i(t\partial_i-x_i\partial_0)$。因此,利用我们之前的选择,很直观的我们有位置:$\hat{x}_i\to x_i$}
\[\text{位置}\hat{x}_i\to x_i \]

我们现在利用算符给出了我们在描述自然的理论中使用的物理量。接下来很合理的我们就需要问:这些算符作用在什么上面?它们是如何与实验中我们做的测量联系起来的呢?我们下一章将会仔细的讨论这件事情,在这里我们只需要知道我们的物理量,也就是算符,是作用在{\it 一些东西}上的。让我们暂时继续用抽象的概念理解算符作用在什么伤。我们叫它$\Psi$,我们后面会探究这个{\it 一些东西}是什么鬼。

现在,我们可以去得到一些非常重要的\mpar{如果你完全不了解量子力学,为什么这些简单地式子会如此重要对你来说会看起来比较奇怪。你也许听说过Heisenberg不确定性原理。在第\ref{sec8.3}节,我们会更进一步的研究量子力学的形式与结构,然后我们就可以看到这个方程告诉我们我们不能够一任意精度同时测量一个粒子的动量和坐标。我们的物理量的解释是算符的测量,而这个方程告诉我们先测量动量再测量位置与先测量位置再测量动量是完全不一样的。}量子力学的方程了。就像上面已经解释过的那样,我们假定我们的算符作用在抽象的$\Psi$上。于是我们就有
\begin{align}
\begin{split}
[\hat{p}_i,\hat{x}_k]&\Psi = (\hat{p}_x\hat{x}_j)\Psi = (\partial_i \hat{x}_j-\hat{x}_j\partial_i)\Psi\\
\underbrace{=}_{\text{莱布尼兹律}}& (\partial_i\hat{x}_j)\Psi + \cancel{\hat{x}_j(\partial_i\Psi)} - \cancel{\hat{x}_j(\partial_i\Psi)} \underbrace{=}_{\text{因为}\partial_i\hat{x}_j = \frac{\partial x_j}{\partial x_i}} i\delta_{ij}\Psi
\end{split}
\end{align}

这个方程对任意的$\Psi$都成立,因为我们没有对$\Psi$做任何假设,因此我们可以直接的写出
\begin{align}
[\hat{p}_i,\hat{x}_j] = i\delta_{ij}
\end{align}

\subsection[自旋与角动量]{Spin and Angular Momentum 自旋与角动量}\label{sec5.1.1}




